
Providing efficient Functions as a Service (FaaS) is challenging due to the serverless programming model and highly heterogeneous and dynamic workloads. 
Great strides have been made in optimizing FaaS performance through scheduling, caching, virtualization, and other resource management techniques.
The combination of these advances and growing FaaS workloads have pushed the performance bottleneck into the control plane itself.
Current FaaS control planes like OpenWhisk introduce 100s of milliseconds of latency overhead, and are becoming unsuitable for high performance FaaS research and deployments.

We present the design and implementation of \sysname, a fast, modular, extensible FaaS control plane which reduces the latency overhead by more than two orders of magnitude.
\sysname~has a worker-centric architecture and introduces a new function queue technique for managing function scheduling and overcommitment. 
\sysname~is implemented in Rust in about 13,000 lines of code, and introduces only 3ms of latency overhead under a wide range of loads, which is more than 2 orders of magnitude lower than OpenWhisk. 


%%% Local Variables:
%%% mode: latex
%%% TeX-master: "paper"
%%% End:
