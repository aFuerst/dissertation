

% \textbf{Performance Gains.}

% \textbf{Why \sysname~?}
\begin{comment}
  \section{Discussion}
There are a number of open-sourced FaaS platforms out there now, so why do we feel the need to make and release yet another?
Nearly all the platforms out there are targeted at end users of FaaS, not built with researchers in mind.
The only one to break this trend is OpenLambda~\cite{hendrickson2016serverless}.
We find it insufficient as it is implemented in Go, whose garbage collection we feel is a significant issue in the latency critical environment of FaaS.
Importantly it lacks both the ability to operate as a cluster and an integrated load generation system, both of which we have implemented both in \sysname~.
% , and has code that is less amenable to research and experimentation.

OpenFaas~\cite{openfaas} and nuclio~\cite{nuclio} both rely on Docker/Kubernetes as their deployment and scaling mechanisms.
These existing tech stacks are highly useful, but limit the research possibilities of a platform, e.g. cold start optimizations and deploying to edge nodes become intractable.
While \sysname~ does have a Docker implementation, it is to showcase the ability implement multiple containerization mechanisms and compare between them.

OpenWhisk~\cite{openwhisk} also relies on a Docker/Kubernetes setup, and has we have shown above has highly unpredictable performance.
The JVM garbage collection, plus high latency variance coming from both their custom platform pieces and third-party CouchDB and Kafka detract from its capability as a research platform.
We have eliminated the third party services from the invocation path, and our design and Rust implementation contribute to the low-overhead low-variance of the platform.
\end{comment}
% A number of other research papers have made one-off serverless systems, but their designs are predicated 

\section{Conclusion}
% \vspace*{-6pt}
\sysname~ a fast, modular, and extensible FaaS control plane. 
It is implemented in Rust in about 13,000 lines of code, and introduces only 3ms of latency overhead under a wide range of loads.
Its worker-centric architecture, resource caching based design, queue-based overcommitment and scheduling, and careful asynchronous implementation, all contribute to low latency and jitter. 

\sysname~ is open source, and intended to serve as a platform for future high-performance FaaS research and deployments.
In the near future, we intend to incorporate support for  Firecracker~\cite{firecracker-nsdi20}  VMs and GPUs; investigate load balancing optimizations; and deploy \sysname~ on HPC and cloud clusters. 

\noindent \textbf{Acknowledgements.}
This research was supported by the NSF grant 2112606. 

%%% Local Variables:
%%% mode: latex
%%% TeX-master: "paper"
%%% End:
