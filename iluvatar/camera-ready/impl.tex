\section{Implementation}

% Use a lot of provided stuff as a starting point
%   Containerd for isolation/container startup
%   CNI for networking
%   prebuilt Docker images
%   Rust language for efficient compiled language, without garbage collection interference
%   Tokio async handling & userspace threads
Our language of choice was Rust, as we wanted to avoid garbage-collected languages and take advantage of the compile-time efficiencies.
The remainder of the system that we wrote ourselves is implemented in $\sim$ 13k lines of Rust.
To manage the expected concurrency for such a system, the Tokio Rust package implements user threads and scheduling that we found very easy to integrate with.

For ease of repeatable and scalable use, we have set up our system to be deployable with Ansible \cite{}.
In a command it can configure, set up, and tear down the applications that make up \sysname{}, and even capture artifacts like logs.
It ties in nicely with the detailed configuration supported by our worker and controller services.
Our services support configuration via a json file, convenient for development and local testing, but less so for large research experiments.
Ansible, plus some clever config loading, allow us to inject arbitrary values as part of a larger command, making deployment for different experiments a breeze.
% With this setup, \sysname{} can be configured and deployed to any number of nodes with a single command.
% With this we easily scripted a variety of setups for our experiments in Section \ref{Evaluation}, giving us consistent and controlled environments and results.
% Workers are configured with a json file on startup, with the various policy options (such as queuing), keep-alive, timeouts, networking, logging, etc.

CNI and container-spec are also provided as json files. For the container-spec, we change....

% Workers have four main services: Containers, invocation, network, and status.
Our workers have four main services that control everything.
The \textbf{ContainerManager} is the arbiter for container tracking as it moves them through as quasi state-machine.
The invoker layer must ask it for a container on which to run an invocation, which is then returned on completion.
On cold starts and prewarm requests, it interacts with backends to create a new container to satisfy a request.
Two concurrent \emph{ContainerPools} are used, tracking for cached and currently-invoking containers respectively.
In the background, it queries the OS for each container's memory usage.
This is handly because functions rarely use most or all of their requested allocation and we can use their real usage in our memory tracking.
If our free memory is not sufficient to start a new container we maintain a list of all alive containers and evict some to make room.
To make evictions on the critical path rare and fast, we have a thread keep the list of containers sorted by eviction priority and ensures a buffer of free memory by eagerly evicting when necessary.

The \textbf{Invoker} controls queuing policies, concurrency limitations, and naturally, actually invoking functions.
It tracks various function characteristics: cold start time, warm execution, IATs, and more; for use in advanced queuing policies.
Rust isn't the ideal language for code reusability, but we've engineered the invoker pipeline such that new queuing policies can be implemented in a few dozen lines of code.
Monitoring invocation patterns for prewarming opportunities, delaying invocations to get a warm start, and the simplest policym FCFS, are all easy to implement.

The \textbf{NetworkManager} creates and caches network namespaces, these are how we provide network isolation between containers.
Caching and maintaining a pool of network namespaces reduces the latency of a cold start, as we show in \ref{};

The \textbf{Status} service queries the other services regularly, recording worker metrics such as number of running functions, containers, memory usage.
Additionally, it captures external values, worker CPU utilization, node load averages, node CPU usage, etc.

% Containermanager is the main component of Containers service. Maintains registered functions, containerpool, system resource usage (CPU and memory).
% Shared across different threads using reference counting.

% \textbf{try-acquire-cores:} is this non-processor-sharing? % ignore this, try-acquire-cores duplicates concurrency limitations, probably will remove as it is redundant

\textbf{Running a function} is a multi-step process that goes through the whole platform. % Describe the flow. 
A request enters \sysname{} and if via the controller it is load-balanced to a worker.
Once at the worker, our invocation is passed to the \emph{Invoker} and put into the queue.
The invoker checks if there is concurrency to run the invocation and removes it from the queue if it can be run.
It must then acquire a container from the \emph{ContainerManager}, ensuring each container only runs one invocation at a time.
At this stage the \emph{Invoker} knows if the invocation will be a cold or warm start and can make a decision based on that information.
A cold start requires the \emph{ContainerManager} to interact with the containerization backend and start a new sandbox.
Once a container is ready, the \emph{Invoker} makes an HTTP call to a server running in the container.
That server then runs the invocation with any arguments passed along.
Finally, the invocation result is propagated back up to the caller.

% Async nature requires a containerlock for running a function in a given container. 
% Invocation is put into the queue, then de-queued into a \emph{different} Tokio runtime to execute.

% Container pool vs. eviction list. Evict random containers of functions? 
% Computing priorities for eviction is done by copying the container list and doing it off-path, since it is triggered only when the free memory drops below the safety limit. 
% Rust ownership makes having a single container-pool without the extra cloning tricky?

% Low-level interfacing with Containerd for creation, destruction, pulling images, etc. 
\begin{comment}
0-702243c7to3354toeabdto38dftoe1dc67894974
['ilúvatar_worker::invoke', 'fcfs_invoke::sync_invocation', 'invoker_trait::enqueue_new_invocation', 'fcfs_invoke::add_item_to_queue', 'invoker_trait::spawn_tokio_worker', 'invoker_trait::invocation_worker_thread', 'invoker_trait::invoke_internal', 'containermanager::acquire_container', 'containermanager::try_acquire_container', 'containermanager::try_lock_container', 'ContainerLock::invoke', 'ContainerdContainer::invoke', 'containerdstructs::update_metadata_on_invoke', 'containerdstructs::call_container', 'containerdstructs::download_text', 'containermanager::return_container']
[0.0, 0.026, 0.039, 0.056, 0.103, 0.132, 0.152, 0.16799999999999998, 0.191, 0.208, 0.294, 0.309, 0.32899999999999996, 0.372, 1.776, 1.867]
[2.15, 2.096, 0.055, 0.02, 0.015000000000000001, 1.7799999999999998, 1.7440000000000002, 0.092, 0.054, 0.014, 1.55, 1.518, 0.017, 1.3639999999999999, 0.032, 0.013]
[1, 2, 3, 4, 3, 4, 5, 6, 7, 8, 6, 7, 8, 8, 8, 6]
['Entry/Enqueued', 'Setup', 'Invoking', 'Returning']
[0.0, 0.103, 0.294, 1.867]
[2.15, 2.0460000000000003, 1.856, 0.28200000000000003]
\end{comment}

\begin{figure}
  \begin{tabular}{|c|c|r|}
    \hline
    Group & Function Name & Time (ms) \\
    \hline
    Ingestion \& Queuing & \begin{tabular}{@{}c@{}}invoke \\ sync\_invoke \\ enqueue\_invocation \\ add\_item\_to\_q \end{tabular} & \begin{tabular}{@{}c@{}}0.026 \\ 0.013 \\ 0.017 \\ 0.02 \end{tabular} \\
    \hline
    Container Operations & \begin{tabular}{@{}c@{}}spawn\_worker \\ dequeue \\ acquire\_container \\ try\_lock\_container \\ \end{tabular} & \begin{tabular}{@{}c@{}}0.029 \\ 0.02 \\ 0.096 \\ 0.014 \\ \end{tabular} \\
    \hline
    Agent Communication & \begin{tabular}{@{}c@{}}prepare\_invoke \\ call\_container \\ download\_result \\ \end{tabular} & \begin{tabular}{@{}c@{}}0.154 \\ 1.364 \\ 0.032 \\ \end{tabular} \\
    \hline
    Returning & \begin{tabular}{@{}c@{}}return\_container \\ return\_results \\ \end{tabular} & \begin{tabular}{@{}c@{}}0.017 \\ 0.266 \\ \end{tabular} \\
    \hline
  \end{tabular}
  \caption{Times spent in differnt parts of the code for a single warm invocation.}
\end{figure}

\textbf{Platform Optimizations}.
We made a number of optimizations that we discovered along the way, often specific to our design.
We switched the queue used by the invoker to be notified on new items, rather than strictly polling for new items.
This both reduced latency (nothing waiting to be discovered), and reduced CPU usage because we can avoid busy waiting.
Rust \emph{async} items are actually resolved via polling, and originally we implemented a polling background thread per-invocation to notify the client thread of completion.
Later this was switched to using a Tokio type for notifications, simplifying our code and improving performance by using framework members.
Caching the HTTP client used to send invocations to containers dramatically improved latencies, as the library uses connection pooling under the hood.
We switched the ContainerPool to use a concurrent dictionary library to reduce contention on a hot data structure.
Lastly, the load generator proved to be a bottleneck source by excessively opening RPC connections.
It multi-plexes connections, making it cheaper to share an existing connection between load threads than it is to open a new connection for each RPC request.
% Optimizations: polling queue -> notification queue, tokio semaphore for invoke complete notification, caching http client for container call, container mgr to dashmap, load gen API to use API factory.



\subsection{Rest} 



Fine-grained performance is obtained using tracing and spans. Filter the output to reduce the overhead. \emph{What did we find using this? Main overhead figure?} . Because spans can be tricky with asynchronous code, we do ....
Finally, worker shutdown entails cleaning all state for repeatable and robust experimentation. 


Containerization overheads were not the target of this work, so we used some well-tested community tools from OCI \ref{} to fulfill that part of the FaaS system.
Containerd \ref{} creates and applies resource cgroups, and handles file system protections for functions.
CNI handles the necessary networking connections and isolation, and finally Docker images integrate seamlessly with Containerd making it easy to create runnable function packages.
% Reusing community-tested tools, Containerd for function isolation, CNI for networking, and Docker images for file systems.

% Modular design
% Configuration allows lots of methods for adjusting internal design
%   Take advantage of polymorphism in Rust for code reusability
%     Easily implement new containerization backends, eviction policies, queue policies, load balancing algorithms, etc.
We had two goals for implementing \sysname{} 1. Make an effective a research platform and 2. Have it produce valid and consistent results.
Researchers must be able to modify and control the system with little effort to efficiently conduct research and test out new ideas.
Additionally, if experiments require lots of hands-on time or vary between iterations, one cannot tell the impact of new ideas.

A high degree of control means plenty of configuration knobs and runtime tunability, of which we have added plenty throughout \sysname{}.
Resource limits for different sized experiments and hardware, metric logging frequencies, and more are provided out of the box.
Making and using code modifications is extremely easy as well, using the polymorphism and code generation capabilities of Rust.
For this work we implemented multiple containerization backends, queuing policies, and load balancers to serve both as comparisons and guides.

We have written our queue policies and load balancers such that they can be written and tested in just a few man-hours and some dozens of lines of code.
Rust's inheritance system limits code reuse, we can't have a parent ''class'' with children using the parent's properties, but you can have the parent implement member functions.
We take advantage of this to move the invocation state management, leaving only the queue ordering and balancing to be implemented.

Container management has been cleanly separated from the isolation mechanisms and their lifecycle.
We can thus compare new isolations, cold start improvements, and more such new research possibilities \emph{on the same hardware}, with just a configuration change and a restart.
New isolation mechanisms naturally take more effort, but can be cleanly dropped-in at runtime taking a couple days of writing and throrough testing.

% All of these major components can then easily be swapped out at runtime via configuration for comparison.
% Our modular code design and configurability can allow these to be swapped out dynamically for custom implementations to facilitate reserach.



% low-variance! and low(ish)-overhead platform
%   No bottlenecks in code
%   Use concurrent datastructures the community provided -> good enough for our needs
%   Mostly R-W locks as write events are rare (registration, cold starts)
%   The mutex lock for the invcation Q has 0 impact under load
Eliminating inconsistencies caused by the Serverless platform are critical for its use in research.
Highly variable results make reasoning with them challenging and one can never be sure if they are from intentional changes or simply background instability.
% Having low variance and low overhead invocations didn't require us to implement needed components ourselves.
Concurrent data structures, from \emph{DashMap}, using sharded read-write locks compose our major pieces, helping us avoid lock contention between threads.
As object creation happens on registration and cold starts, which are uncommon, we continue to use read-write locks elsewhere to avoid contention on global objects.
The primary "bottleneck" comes from our enqueueing of every invocation into a global queue protected by a basic mutex.
Our results below show that even handling many thousands of requests per second, it is no worse than a queue-less implementation \ref{}.

% Low CPU usage under load
%    ~ 20% of one CPU core
The efficient nature of compiled Rust code and our implementation moderates the CPU usage of the worker process.
Keeping this low is important to avoid interference on functions and minimize overhead.
Even under a heavy and sustained load that fills our servers \ref{}, the \sysname{} worker process never uses more than 20\% of a CPU core.

% Load Generator
% Enables live testing and simulation setup
%   Both controller (cluster) and individual worker
% scaling - test throughput ability with increasing number of clients
% mention this issue?: tput breakdown with increasing number of clients if they keep opening new RPC requests
%       'clone' of RPC connection from tonic scales better than establishing new connection
%       means that system will not scale with large % of clients coming from a varietey of sources (probably takes a lot though)
% benchmark - run a function on an empty setup to capture its performance characteristics
%   Cold, pre-warmed, & warmed setups
%   useful for later evauluation of system under stress
% trace - run a pre-generated trace against the system
%   open-loop load generator, invocation happens at a pre-determined time
%   Captures all the invocation details: tid, execution duration, e2e latency, etc.
%   Pre-warms containers based on function characteristics: IAT, expected runtime characteristics, etc.
%       Handles the "warm state" of a system to avoid herd of cold container creation
%   Able to use benchmark data to "map" custom functions to those in trace
%       Or specify functions to use, or use lookbusy for "realistic" usage
%       Not perfect, if under CPU time sharing

Our load generator is closely coupled with the platform, being able to target a cluster fronted by the controller, or a single worker.
At it's simplest we can test the running characteristics of a variety of functions, which we used to inform both our design and debugging directions.
We then leverage those function metrics to execute realistic loads against the system under test.
For deciding what user code is actually executed on workers, we have provided several possibilities.
The function characteristics from above are used to intelligently ''map'' ported FaasBench \cite{} functions to the trace being run, so the running times are similar.
Supplementing this, users can either specify code to be run in each function or utilize \emph{lookbusy} \cite{} for wall-clock accurate, cpu-using, and memory-usage matching invocations.

% Trace generation
% Use the azure stuff
% Can generate traces a few diff ways: iat, ecdf, actual invokes
% function sizes based on # of functions taken from the Azure trace or expected # of outstanding requests per second (little's law)
% Any hand-rolled trace with specific functions and IATs for testing and creating specific scenarios: e.g. big burst

To generate loads, we pull real-world data from pub Azure \cite{} datasets.
\sysname{} can generate basic traces using the mean inter-arrival times of functions, such that functions will appear with consistent frequency, plus some Gausian noise distribution.
% and even directly replicate invocation patterns.
The most advanced are traces use ECDFs \cite{}, where the range of IATs of a function form a sample distribution to when it will be invoked next.
This allows us to capture the bursty nature of many functions, having alternating periods of high and low invocation frequencies.

% Mention energy monitoring capabilities?
We have integrated additional systemreporting capabilities for other novel FaaS research as well.
\sysname{} can monitor a variety of hardware and software energy metrics to compute platform and per-function and per-invocation energy contributions \cite{}.
With this capability, the platform enables new billing, scheduling, and load balancing opportunities for serverless.

%%% Local Variables:
%%% mode: latex
%%% TeX-master: "paper"
%%% End:
