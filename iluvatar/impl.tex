\section{Implementation}
\label{sec:impl}

% Use a lot of provided stuff as a starting point
%   Containerd for isolation/container startup
%   CNI for networking
%   prebuilt Docker images
%   Rust language for efficient compiled language, without garbage collection interference
%   Tokio async handling & userspace threads

\sysname~ is implemented in Rust in about 13,000 lines of code.
It will be open-sourced upon paper acceptance. 
Its low latency and lack of jitter are attributable to the various low-level profile-guided performance optimizations we have implemented during the course of its development and testing.
Function handling and container management in the worker make up a majority of the implementation footprint and focus. 
Ours is a heavily asynchronous implementation using the \texttt{tokio} library in Rust, and various function lifecycle events spawn new userspace threads and trigger callbacks. 
The major data structure shared by the various worker threads is the container pool, which is implemented using the \texttt{dashmap} crate, which is a concurrent associative hashmap--- this provides noticeable latency improvements compared to a mutex or read-write lock.
Conversely, we still use a mutex for the queue, since we found minimal performance degradation compared to a no-queue architecture during profiling. 
These, and many other small optimizations, keep the \sysname~
resource consumption small: even under a heavy and sustained load that saturates a 48 CPU server, the worker process uses less than 20\% of a single CPU core. 




%%%%%%%%%%%%%%%%%%%%%%%%%%%%%%

%\noindent \textbf{Agent Optimization.}
%All our functions are written Python, and we do some Python-specific optimizations during their prewarm.
%The agent imports the function Python code, loading it and any library dependencies into memory, plus executing statements not in functinos, such as downloading an ML model.


% \paragraph{Deployment}
% For ease of repeatable and scalable use, we have set up our system to be deployable with Ansible \cite{}.
% In a command it can configure, set up, and tear down the applications that make up \sysname{}, and even capture artifacts like logs.
% It ties in nicely with the detailed configuration supported by our worker and controller services.
% Our services support configuration via a json file, convenient for development and local testing, but less so for large research experiments.
% Ansible, plus some clever config loading, allow us to inject arbitrary values as part of a larger command, making deployment for different experiments a breeze.

% With this setup, \sysname{} can be configured and deployed to any number of nodes with a single command.
% With this we easily scripted a variety of setups for our experiments in Section \ref{Evaluation}, giving us consistent and controlled environments and results.
% Workers are configured with a json file on startup, with the various policy options (such as queuing), keep-alive, timeouts, networking, logging, etc.

% CNI and container-spec are also provided as json files. For the container-spec, we change....

% Workers have four main services: Containers, invocation, network, and status.

% \vspace*{-8pt}
\subsection{Support for FaaS research}
\label{sec:impl:support}

One of our major design goals is for a reliable and extensible control plane for performance-focused FaaS research.
We now describe some of the \sysname~ features and our experiences in extending it.


\noindent \textbf{Performance Metrics.}
We keep track of all internal and external function metrics (such as their cold/warm execution time histories, inter arrival times, memory footprints, etc.) and provide them to all components of the control plane, and also to external services.
%
One of \sysname's implementation goals was to reduce the reliance on external services for system monitoring etc.
We thus track key system metrics like CPU usage, load averages, and even CPU performance counters and system energy usage using RAPL and external power meters.
These metrics are collected using async worker threads, and provide a single consistent view of the system performance.
%
Additionally, we also use and provide Rust-function tracing for fine-grained performance logging and analysis.
We use the \texttt{tracing} crate to instrument the passage of invocations through the control plane components, and obtain detailed function level timing information, which is used for identifying control plane and container-layer bottlenecks. 
%These logged details helped us with performance bottlenecks, but by default we keep disabled because it can increase per-invocation latency due to the overheads and CPU contention from creating so many logging events and it increases log file output size dramatically. 

\noindent \textbf{Adding New Policies and Backends.}
%
Using function and system metrics allows for easy development of data and statistical learning based resource management policies to be implemented.
Our baseline policy implementations for keep-alive eviction, queuing, load-balancing, are all easily extensible using Rust traits, polymorphism, and code generation. 
In our experience, adding new policies is relatively straight-forward, even for new-comers.
For example, all the priority-based queuing policies (SJF, EEDF, RARE, etc.) were implemented by extending the base FCFS policy.
Implementing and testing these policies took less than a few dozen lines and about four hours for a graduate student unfamiliar with the code-base. 

The default container runtime backend is \texttt{containerd}, but the interface is small, and supporting new backends is relatively easy.
We added Docker support in about 400 lines and one person-day of development effort.


\textbf{Load-generation and Testing.}
In the spirit of providing a full-featured system for FaaS experimentation, we have developed a load-generation framework.
It can do closed and open loop load generation, and be parameterized by the number and mixture of functions, their IAT distributions, etc. 
The testing framework can use functions from FaaS suites like FunctionBench~\cite{kim2019functionbench}, or custom sized functions that run lookbusy~\cite{lookbusy} for generating specific CPU and memory load.
%
The open-loop generation produces a timeseries of function invocations, which is helpful for repeatable experiments.
The functions' IAT distributions can be exponential, or be derived from empirical FaaS traces like the Azure trace~\cite{shahrad_serverless_2020}.


For the Azure trace, we start by randomly sampling functions and computing the CDF of their IATs.
We compute the expected load level in the system using Little's law, by finding the expected number of concurrent invocations for each function and adding them for all functions.
This expected load can be significantly different from the capabilities of the system under testing (for example, 100 concurrent functions will overload a 12 core system). 
Therefore, we can scale the individual function IAT CDFs to find a suitable load. This also allows us to change the relative popularities of individual functions, and conduct fine-grained sensitivity experimentation (like examining system performance when the popularity of one single function
changes, etc.).
We can generate larger traces by layering, and merging the traces from multiple smaller workloads.  


For synthetic functions (using lookbusy), we use their distribution of running times and memory consumption when generating the workload.
When using real functions from a benchmark-suite like FunctionBench, for each randomly sampled function, we use its average execution time (from the full trace), and assign it the closest function in the suite.
For example, if the average running time of a candidate function in the Azure trace is 8 seconds, we represent it using the ML-training function, which has the closest running time of 6 seconds.




%%%%%%%%%%%%%%%%%%%%%%%%%%%%%%


%%% Local Variables:
%%% mode: latex
%%% TeX-master: "paper"
%%% End:
