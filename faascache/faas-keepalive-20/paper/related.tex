\label{sec:related}
\vspace*{\subsecspace}

% \section{Related Work}

\noindent \textbf{Function Keep-alive.}
Mitigating cold starts is one of the central performance problems in FaaS, and has received commensurate attention in both academia and industry.
%
The initialization or startup time of functions can be reduced  by reducing container startup overheads~\cite{oakes_sock_2018,mohan_agile_2019, akkus_sand_2018}, or deploying functions inside ultra-light containers, VMs, or unikernels~\cite{unikernels,firecracker-nsdi20}.
%
While these approaches can reduce the cold start overhead associated with the virtual environment creation, the other sources of overheads remain, such as losing all application initialized variables, cached files, etc.
As we have shown, keep-alive essentially serves the role of caching, and fast startup only reduces the ``miss'' penalty, and does not eliminate it.


Principled keep-alive policies for functions have recently gained attention: the recent dataset and policy from the Azure function trace~\cite{shahrad_serverless_2020} shows the importance and effectiveness of keep-alive policies. 
In contrast to our work, their policy does not take the function size into consideration, and uses a time-series predcition approach (effectively capturing recency and frequency), and combines it with a predictive ``prefetching'' approach. 
As we have shown, function sizes are a crucial characteristic, and the use of caching allows the use of advanced analytical and modeling approaches for serverless computing in general. 
%
Earlier work has focused on simple ``warm container pools''~\cite{lin_mitigating_2019}, or using polling to keep cloud functions warm~\cite{warm2,warm1}. 
%
Our work considers functions individually---function scheduling with DAG based approaches~\cite{carver_search_2019} is effective for function-chains, and are orthogonal and complementary to our work. 
%
Hiding function latency using data caching (such as redis) for database applications is investigated in~\cite{ghosh_caching_2019}. 
The ENSURE~\cite{ensure_acsos20} system handles keep-alive and resource provisioning for CPU resources using queuing theory techniques.
Our focus is on memory-constrained keep-alive and provisioning, and CPU-focused approaches are complementary to our work. 

\begin{comment}
\noindent \textbf{Caching.}
Our choice of Greedy-Dual-Size-Frequency was motivated by the ease with which its parameters mapped to function keep-alive, in particular the frequency vs. size tradeoff.
Cache eviction algorithms have a long history, although the focus has predominantly on uniform sized objects (such as disk blocks, RAM pages, cache lines, etc.).
Certain caching optimizations relying on spatial locality (such as look-ahead and ARC) are not directly applicable to keep-alive. 
The size-aware cache algorithms and models used for web-caches, proxies, and CDNs form the basis of our work. 
Cache provisioning using hit/miss ratio curves is a common approach, and these curves can also be constructed dynamically, which is part of our future work.
%
\end{comment}

\begin{comment}

The Azure function traces and keep-alive approach 

Prior work has investigated reducing container startup overheads, which can reduce the 

\noindent \textbf{Usecases}

~\cite{cartas_reality_2019} questions the need for edge computing for ML inference, because of the decreasing inference latency and costs, and network latency becomes the primary bottleneck.

Berkeley view~\cite{jonas2017occupy, jonas_cloud_2019}

Interactive applications---RunBox ~\cite{glikson_runbox_2019}.

Data analytics with ServerMix~\cite{garcia-lopez_servermix_2019}

ExCamera

PyWren

Serverless analytics with Flint wtf

From laptop to lambda ~\cite{fouladi_laptop_2019}

Monolithic to serverless with Ripple~\cite{joyner_ripple_2020}

HPC~\cite{mocskos_faaster_2018}

SPEC group report~\cite{van_eyk_spec_2017}

\noindent \textbf{cold start problem}

~\cite{lin_mitigating_2019} is a project report that uses ML inference, webserver workloads as we are. Knative platform. Maintain a pool of warm containers. No real policy though.

HyperFaas claims to address cold start problem? ~\cite{zhang_hyperfaas_2019}. 

\noindent \textbf{Keep-alive Mechanisms.} Most closely related work goes here.. 

\noindent \textbf{Frameworks}

Cirrus~\cite{carreira_cirrus_2019} does end to end ML model training, hyperparam tuning on a new serveless framework. Not inference though?

SAND~\cite{akkus_sand_2018} is based on openwhisk, but focused more on the inter function messaging?

Wukong~\cite{carver_search_2019} has an interesting DAG scheduling solution---decentralized scheduling. Evalutes on numerical computing problems and compares with Dask (Python).

SPOCK~\cite{gunasekaran_spock_2019} combines VMs and serverless functions. 

Named functions as a service~\cite{krol_nfaas_2017}---uses unikernels!!
More work along similar lines~\cite{lin_computation_2019,mtibaa_towards_2018,xylomenos_named_2019,krol_computation_2018}


Peeking behind the curtains

SOCK~\cite{oakes_sock_2018}

OpenLambda~\cite{hendrickson2016serverless}

GrandSlam ~\cite{kannan_grandslam_2019} ~\cite{} tries to provide SLA guarantees for microservices.

Performance comparison by Geoffrey~\cite{lee_evalution_2018}


\noindent \textbf{Resource management}

Auction based function placement in serverless~\cite{bermbach_towards_2019}.

Checkpointing and migration with CRIU for IoT devices in~\cite{karhula_checkpointing_2019}. What's the motivation with stateless functions? Nothing---they focus on stateful functions.


``Living on the edge''~\cite{kulkarni_living_2019} explores transaction support and fault tolerance for stream processing. 


Data management (really caching?) with the Locus shuffling system~\cite{pu_shuffling_2019}. 


Caching strategies for data in~\cite{ghosh_caching_2019}

More storage optimizations?~\cite{zhang_narrowing_2019}

\noindent \textbf{Formal }

~\cite{hall_execution_2019} provides an architecture with WASM on the edge.

~\cite{riis_nielson_no_2019} provides a serverless kernel calculus combining lambda calculus for computation and pi-calculus for communication.

\end{comment}



%%% Local Variables:
%%% mode: latex
%%% TeX-master: "paper"
%%% End:
