% \usepackage[english]{babel}
% \usepackage[utf8x]{inputenc}
% \usepackage{hyperref}
% \usepackage[protrusion=true,expansion=true]{microtype}
% \usepackage{amsmath,amsfonts,amsthm}     % Math packages
% \usepackage{graphicx}                    % Enable pdflatex
% \usepackage[svgnames]{xcolor}            % Colors by their 'svgnames'
% \usepackage{geometry}
% \usepackage{url}
% \usepackage{multicol}
% \usepackage{etaremune}
% \usepackage[none]{hyphenat}%%%%
% \usepackage{comment}
% \usepackage{enumitem}

% %%% Custom sectioning (sectsty package)
% %%% ------------------------------------------------------------
% \usepackage{sectsty}


%%% Macros
%%% ------------------------------------------------------------
\newcommand{\sysname}{Il\'{u}vatar} % \xspace}

\newcommand{\phone}{(440) - 669 - 5865}
\newcommand{\email}{\url{fuersta.2013@gmail.com}}
\newcommand{\linkedin}{\url{linkedin.com/in/alex-fuerst}}
\newcommand{\github}{\url{github.com/aFuerst}}
\newcommand{\website}{\url{afuerst.github.io}}

\newlength{\spacebox}
\settowidth{\spacebox}{8888888888}			% Box to align text
\newcommand{\sepspace}{\vspace*{0 mm}}		% Vertical space macro

\newcommand{\MyName}[1]{ % Name
        \Huge \usefont{OT1}{lmr}{b}{n} \hfill \centerline{#1}
        \par \normalsize \normalfont}
        
\newcommand{\NewPart}[1]{
\vspace*{-.25cm}
\section*{\uppercase{#1}}}

\newcommand{\SkillsEntry}[2]{      % Same as \PersonalEntry
        \noindent\hangindent=2em\hangafter=0 % Indentation
        \parbox{\spacebox}{        % Box to align text
        \textit{#1:}}			   % Entry name (birth, address, etc.)
        \hspace{1.5em} #2 
        \normalsize \par}    % Entry value	

\newcommand{\WorkEntry}[6]{				  % Same as \EducationEntry
        \noindent \textit{#4} \hfill \textit {#5}   \par       % Company % Location
        \noindent \textbf{#1} \hfill      % Jobname
        \vspace*{-.1cm}
        {#2} - {#3} \par
        \noindent\hangindent=2em\hangafter=0 \small #6 % Description
        \normalsize \par}

\newcommand{\WorkName}[2]{
        \noindent \textit{#1} \hfill \textit {#2}   
        \normalsize \par       % Company % Location
        }

\newcommand{\PositionEntry}[2]{
        \noindent \textbf{#1} \hfill      % Jobname
        {#2} \normalsize  \par  % Duration
}

\newcommand{\JobDesc}[1]{
    \noindent\hangindent=2em\hangafter=0 \small #1 \normalsize % Description
}

\newcommand{\AwardEntry}[3]{
    \noindent \textit{#1} % institution
    \textbf{#2}
     \hfill	\text{#3}				% name
     \normalsize \par}

\newcommand{\ProjectEntry}[2]{
        \noindent \textbf{#1} \par      % Title
        \noindent\hangindent=2em\hangafter=0 \small #2% Description
        \normalsize \par}

\newcommand{\CourseworkEntry}[2]{
    \noindent \textbf{#1} \par      % Title
    \noindent\hangindent=2em\hangafter=0 \small #2% Description
    \normalsize \par}

\newcommand{\PresentationEntry}[4]{
    \noindent #1. % Name
    #2. % Conference & date
    \href{#3.}{Slides} % Slides URL
    \href{#4.}{Video} % Presentation video URL
    \normalsize \par \vspace{2pt}}

\newcommand{\PresentationEntryNoVid}[3]{
      \noindent #1. % Name
      #2. % Conference & date
      \href{#3.}{Slides} % Slides URL
      \normalsize \par \vspace{2pt}}
        
\newcommand{\EducationEntry}[5]{
  \noindent \textbf{#1}     % Study
  \hfill {#2} \par
  \textit{#3} \hfill {#4}	%school  % award
  \noindent\hangindent=2em\hangafter=0 \small #5% Description
  \normalsize}

\newcommand{\PresentationList}[1]{
  \noindent\hangindent=3em\hangafter=0  \small Presentations: \textit {#1} % List
  \hfill \normalsize}

  \newcommand{\CourseList}[1]{
  \noindent\hangindent=3em\hangafter=0  \small Courses: \textit {#1} % List
  \hfill \normalsize}
  
\newcommand{\ObjectiveStatement} {
\noindent Systems engineer with five years' development experience, expertise in low-level virtualization, operating systems, and high-level distributed systems, who can analyze and communicate findings to general audiences.
Looking to design solutions to the inefficiencies and problems of modern computing in cloud-scale environments.
}

\newcommand{\Education} {
  \NewPart{Education}

  \EduPhd

  \EduMs
  
  \EduBs 
  
  \EducationEntry{Diploma with Honors}{May 2013}{Medina High School}{GPA 3.6}  

}

\newcommand{\EduPhd} {
  \EducationEntry{Computer Engineering, PhD}{Expected May 2024}{Indiana University, Intelligent Systems Engineering}{Major GPA 3.9}  
}
\newcommand{\EduMs} {
  \EducationEntry{Computer Engineering, MS}{Dec 2023}{Indiana University, Intelligent Systems Engineering}{Major GPA 3.9}
}
\newcommand{\EduBs} {
  \EducationEntry{Computer Science, Bachelor of Science}{May 2017}{Xavier University}{Major GPA 3.6}
}

\newcommand{\MyAuth}[1]{\textbf{#1}}
\newcommand{\PublicationEntry}[3]{
    \noindent #1. % Authors
    #2. % Title
    #3 % Conference & date
    \normalsize \par \vspace{2pt}}
\newcommand{\Conference}[2]{
\textit{#1}. Acceptance Rate = #2\%}
\newcommand{\ShortConference}[1]{
\textit{#1}.}
\newcommand{\HPDC}[2] {\Conference{International ACM Symposium on High-Performance Parallel and Distributed Computing [HPDC] #1}{#2}}
\newcommand{\ASPLOS}[2] {\Conference{International Conference on Architectural Support for Programming Languages and Operating Systems [ASPLOS] #1}{#2}}
\newcommand{\HpdcShort}[1] {\ShortConference{Symposium on High-Performance Parallel and Distributed Computing #1}}
\newcommand{\AsplosShort}[1] {\ShortConference{Conference on Architectural Support for Programming Languages and Operating Systems #1}}

\newcommand{\Publications} {
  \NewPart{Publications}
  \begin{etaremune}
    \setlength\itemsep{0em}
    \item \PublicationEntry{\MyAuth{Alexander Fuerst}, Abdul Rehman, and Prateek Sharma}
    {\sysname~: A Fast Control Plane for Serverless Computing}
    {\HPDC{2023}{21}}

    \item \PublicationEntry{\MyAuth{Alexander Fuerst}, and Prateek Sharma}
    {Locality-aware Load-Balancing For Serverless Clusters}
    {\HPDC{2022}{19}}

    \item \PublicationEntry{\MyAuth{Alexander Fuerst}, Stanko Novakovic, Inigo Goiri, Gohar Irfan Chaudhry, Prateek Sharma, Kapil Arya, Kevin Broas, Eugene Bak, Mehmet Iyigun, and Ricardo Bianchini}
    {Memory-Harvesting VMs in Cloud Platforms}
    {\ASPLOS{2022}{20}}

    \item \PublicationEntry{\MyAuth{Alexander Fuerst}, and Prateek Sharma}
    {FaasCache: Keeping Serverless Computing Alive With Greedy-Dual Caching}
    {\ASPLOS{2021}{18.8}}

    \item \PublicationEntry{\MyAuth{Alexander Fuerst}, Ahmed Ali-Eldin, Prashant Shenoy, and Prateek Sharma}
    {Cloud-scale VM-deflation for Running Interactive Applications On Transient Servers}
    {\HPDC{2020}{22}}
  \end{etaremune}  
}

\newcommand{\ResearchExpEntry}[2]{
        \noindent \textbf{#1} \par      % Title
        \vspace*{-.0cm}
        \noindent\hangindent=2em\hangafter=0 \small #2% Description
        \normalsize \par}

% \ResearchExpEntry{Serverless Scheduling and GPU Acceleration}
% {Designed a load balancing algorithm to optimize locality while preventing worker overloading, providing 2x reduction in latency.
% Created a first of its kind caching policy in FaaS 
% Unique 
% Published at HPDC 2023, ASPLOS 2021.}

\newcommand{\IluvatarExp} {
\ResearchExpEntry{\href{https://github.com/cos-in/iluvatar-faas}{\sysname~Serverless Control Plane}}
{An open-source, fast, jitter-free Serverless control plane written in $\sim$23k lines of Rust.
\sysname~gives 100x reduction in p99 latency under high load over competing open-source platforms.
Combines distributed systems, scheduling, caching, and virtualization techniques to achieve performance.
Published at \href{https://afuerst.github.io/assets/Il\%C3\%BAvatar.pdf}{HPDC 2023}.}}

\newcommand{\FaaSCtrlPlaneExp} {
\ResearchExpEntry{\href{https://github.com/cos-in/iluvatar-faas}{\sysname~Control Plane \& GPU Acceleration for FaaS}}
{An open-source, fast, jitter-free Serverless control plane written in $\sim$23k lines of Rust.
\sysname~gives 100x reduction in p99 latency under high load over competing open-source platforms.
% Combines distributed systems, scheduling, and virtualization techniques to achieve performance.
Enabled control plane management of GPU memory to reuse GPU function containers and improve scheduling for 14x better latency.
Published at \href{https://afuerst.github.io/assets/Il\%C3\%BAvatar.pdf}{HPDC 2023}.}}

\newcommand{\FaaSExp} {
\ResearchExpEntry{Scheduling and Mitigating Cold Starts in Serverless Clusters}
{Designed a load balancing algorithm to optimize cluster locality and avoid worker overloading, providing 2x latency reduction.
Added intelligence to systems to handle the bursty and heterogeneous traffic ubiquitous in FaaS.
Created container caching strategy to reduce resource usage by 3x in busy workers.
Published at \href{https://afuerst.github.io/assets/FaasCache.pdf}{ASPLOS 2021}, \href{https://afuerst.github.io/assets/lbfaas.pdf}{HPDC 2022}.}}

\newcommand{\FaasLBExp} {
\ResearchExpEntry{Locality and Load-Balancing for Serverless Clusters}
{Designed a drop-in replacement load balancing algorithm that optimizes for locality while preventing worker overloading.
Load balancing algorithm provided a 2x reduction in function latency over the existing algorithms.
It can handle bursty or heterogeneous traffic and supports cluster auto-scaling under high load. 
Published at \href{https://afuerst.github.io/assets/lbfaas.pdf}{HPDC 2022}.}
}

\newcommand{\DeflationExp} {
\ResearchExpEntry{Cloud-scale VM Deflation}
{Used hypervisor mechanisms to create deflatable (dynamically resourced) VMs, to reduce cluster underutilization.
Showed that microservices inside such VMs have negligible overhead and can adjust to resource changes.
Examined policies to allow 50\% resource overcommitment while keeping cluster performant.
Published at \href{https://afuerst.github.io/assets/deflation.pdf}{HPDC 2020}.}}

\newcommand{\CloudResourceExp} {
\ResearchExpEntry{Cloud Cluster Resource Utilization}
{Improved para-virtual interface between host \& guest for better dynamic resource allocation support.
% Used hypervisor mechanisms to create deflatable (dynamically resourced) VMs, to reduce cluster underutilization.
% Showed that microservices inside such VMs have negligible overhead and can adjust to resource changes.
% Examined policies to 
Designed cooperative policies to allow 50\% resource overcommitment with minimal application interference.
Improved memory utilization inside Azure by 25\% with no impact to guest VMs.
Published at \href{https://afuerst.github.io/assets/deflation.pdf}{HPDC 2020} \href{https://afuerst.github.io/assets/Memory-Harvesting.pdf}{ASPLOS 2022}.}}


\newcommand{\GoogIntern} {
  \WorkEntry{Software Engineering Intern}{May 2023}{August 2023}{Google, Inc.}{Mountain View, California}{
    \begin{itemize}
    \setlength\itemsep{0em}
    \item Delved into performance of advanced Linux and KVM-based virtualization technologies under cloud workloads
    \item Explored techniques to seamlessly upgrade VMM and hypervisor with zero downtime to guest OS and applications
    \item Modified Linux kernel, KVM, and Cloud Hypervisor VMM to test possibilities for seamless upgrade
    \item Developed proof-of-concept experiments to show feasibility of designed techniques
    \end{itemize}
    }
}
\newcommand{\GoogCloudIntern} {
  \WorkEntry{Software Engineering Intern}{May 2023}{August 2023}{Google, Inc.}{Mountain View, California}{
    \begin{itemize}
    \setlength\itemsep{0em}
    \item Delved into performance of advanced Linux and KVM-based virtualization technologies under cloud workloads
    \item Explored techniques to seamlessly upgrade VMM and hypervisor with zero downtime to guest OS and applications
    \item Conferred with multiple teams on design of how conduct zero-downtime upgrades of hosts with VMs 
    \item Modified Linux kernel, KVM, and Cloud Hypervisor VMM in proof-of-concept tests for designs
    % \item Developed proof-of-concept experiments to show feasibility of designed techniques
    \end{itemize}
    }
}
\begin{comment}
At Google I worked on advanced KVM-based virtualization technologies with their internal Linux kernel memory management team, under the supervision of Pasha Tatashin. 
We examined the causes of virtual machine guest exits to the host hypervisor, and ways of minimizing such exits. 
Then we designed techniques to seamlessly upgrade the hosting virtual machine monitor and hypervisor without interruption of the running guest.
I developed proof-of-concept experiments by modifying the Rust-based Cloud Hypervisor as VMM, KVM as hypervisor, and Linux as the guest OS to show feasibility of our techniques.
\end{comment}


\newcommand{\CloudMsftIntern} {
  \WorkEntry{Research Intern}{May 2021}{August 2021}{Microsoft Research}{Redmond, Washington}{
  \begin{itemize}
  \setlength\itemsep{0em}
  \item Analyzed modern hypervisor and control plane's performance under strenuous runtime conditions
  % \item Modified cloud control plane, hypervisor, and guest OS to improve cluster memory management
  \item Enhanced production Azure control plane with active memory management capabilities
  \item Collaborated with Azure team to alleviate production contentions and plan rollout across clusters
  \item Improved resource utilization by 20\% in Azure without impact to hosted virtual machines or applications
  % \item First author on published research paper that resulted from this work
  \end{itemize}
  }
}
\begin{comment}
My internship at Microsoft Research was targeted at improving physical memory utilization inside Microsoft Azure. 
The goal was to enable running low-priority virtual machines (VMs) using the extra resources, but be able to quickly transfer the resources to regular VMs when they were started.
To accomplish this, we modified both the hypervisor and guest OS improving guest VM memory adjustment performance.
We also created a para-virtual interface to optimize guest OS and internal application's runtime performance and ability to adjust to memory pressure.
These resulted in improved resource utilization in Azure by 20% with no impact to hosted VMs or applications.
All of this work culminated in a research paper in which I was the first author, and then presented by me at ASPLOS 2022 in Switzerland.
\end{comment}
\newcommand{\MsftIntern} {
  \WorkEntry{Research Intern}{May 2021}{August 2021}{Microsoft Research}{Redmond, Washington}{
  \begin{itemize}
  \setlength\itemsep{0em}
  \item Analyzed modern hypervisor and control plane's performance under strenuous runtime conditions
  % \item Modified hypervisor and guest OS to improve guest VM memory resizing
  \item Modified Azure control plane, hypervisor, and guest OS to improve cluster virtual memory management by 50\%
  \item Collaborated with Azure team to alleviate production contentions and plan rollout across clusters
  \item Improved resource utilization by 20\% in Azure without impact to hosted virtual machines or applications
  % \item First author on published research paper that resulted from this work
  \end{itemize}
  }
}

\newcommand{\Hyland} {
  % \PositionEntry {Developer 1}{2017-2018}
  % \PositionEntry {Developer 2}{2018-2019}
  \WorkEntry{Developer 1 \& 2}{June 2017}{July 2019}{Hyland Software}{Westlake, Ohio}{
  \begin{itemize}
  \setlength\itemsep{0em}
  \item Developed features and wrote tests for a cloud application capable of handling thousands of daily users
  \item Troubleshot complex issues of a multiservice .Net SaaS application running in production
  \item Refactored monolith application into a microservice design and support autoscaling inside Kubernetes
  \item Modernized application CI/CD pipeline to halve time-to-deployment for features and enable rollback scenarios
  \end{itemize}
  }
}
\begin{comment}
I was a member of the backend team developing a cloud application capable of handling thousands of users.
Daily tasks included developing new features, fixing bugs found by QA or development, and troubleshooting complex issues of the multi-service application while it was running in production.
My role evolved into a mixed developer/DevOps role where we planned and designed the breakup of our monolith codebase to run as a series of microservices in Kubernetes.
I was also responsible for modernizing our CI/CD pipeline which halved time-to-deployment for features and releases, and allowed easy rollback scenarios.
\end{comment}
\newcommand{\RaIU} {
  \WorkEntry{Research Assistant}{August 2021}{Present}{Indiana University}{Bloomington, Indiana}{
  \begin{itemize}
  \setlength\itemsep{0em}
  % \item Identified opportunities for improving cloud computing platforms and distributed systems
  % \item Developed novel cutting-edge techniques and algorithms to improve the state-of-the-art for computer systems
  \item Performed advanced research in cloud resource management, serverless computing, and virtualization
  \item Developed novel techniques and algorithms improving resource utilization and latency in cloud control planes
  \item Designed experiments to showcase techniques' effectiveness and transform results into actionable data
  \item Published several first author papers and gave presentations at high-impact conferences
  \end{itemize}
  }
}
\begin{comment}
The latter half of my PhD was as a dedicated research assistant under my advisor.
We would spend time identifying opportunities for enhancing existing systems or potential for entirely new designs. 
Once a direction was decided, we prototyped our plans in simulation form and implemented them as real-world systems.
Each new work required planning and executing on a variety of experiments to showcase their effectiveness. 
My research has spanned from cloud resource management, serverless computing, virtualization, and simulation acceleration.
These works have resulted in 5 published first author papers and presentations of those papers at several top conferences.
\end{comment} 
\newcommand{\AiIU} {
  \WorkEntry{Associate Instructor}{August 2019}{May 2020}{Indiana University}{Bloomington, Indiana}{
    \begin{itemize}
    \setlength\itemsep{0em}
    \item Assisted with Engineering Cloud Computing \& Distributed Computing Engineering course work
    \item Created assignments and exams given to students
    \item Hosted lab and office hours to discuss project design and assist with student questions
    \end{itemize}
    }
}

\newcommand{\WorkExperience} {
  \NewPart{Experience}

  \GoogIntern
  \sepspace
  
  \MsftIntern
  \sepspace
  
  \RaIU
  \sepspace

  \AiIU
  \sepspace
    
  \Hyland
  \sepspace
  
  \WorkEntry{Teaching Assistant}{August 2016}{December 2016}{Xavier University} {Cincinnati, Ohio}{
    \begin{itemize}
    \setlength\itemsep{0em}
    \item Work with students during class exercises
    \item Host office hours answering questions and giving guidance on assignments
    \end{itemize}
    }
  \sepspace
  
  \WorkEntry{Student Technician Tier II}{August 2013}{August 2017}{Xavier University} {Cincinnati, Ohio}{
  \begin{itemize}
  \setlength\itemsep{0em}
  \item Troubleshoot complex technology issues and provide onsite service and repair for faculty, staff and public computing
  \item Provide software, hardware and network problem resolution
  \item Handle tickets escalated from Tier I
  \end{itemize}
  }
  \sepspace
  
  \WorkEntry{NSF REU Researcher}{May 2016}{August 2016}{Salisbury University}{Salisbury, Maryland}{
  \begin{itemize}
  \setlength\itemsep{0em}
  \item Applied emerging parallel computing models using GPU and CPU parallelism with NVIDIA's CUDA
  \item Tackled data and compute-intensive problems in geographic information systems
  \item Presented findings to GIS professionals and Salisbury Faculty
  \end{itemize}
  }
  \sepspace
  
  \WorkEntry{Paralegal Intern}{August 2012}{June 2013}{Critchfield, Critchfield \& Johnston, Ltd.}{Medina, Ohio}{
  \begin{itemize}
  \setlength\itemsep{0em}
  \item Prepared and delivered documents to county offices
  \item Finalized legal binders for delivery to clients
  \end{itemize}
  }  
}

\newcommand{\Presentations} {
  \NewPart{Presentations}

  \begin{itemize}
    \setlength\itemsep{0em}

    \item \PresentationEntryNoVid{\sysname~: A Fast Control Plane for Serverless Computing}
    {HPDC 2023}
    {https://afuerst.github.io/assets/iluvatar-presentation.pptx}

    \item \PresentationEntry{Locality-aware Load-Balancing For Serverless Clusters}
    {HPDC 2022}
    {https://afuerst.github.io/assets/faas-lb-presentation.pptx}
    {https://youtu.be/nEB45\_dtx6U}

    \item \PresentationEntry{Memory-Harvesting VMs in Cloud Platforms}
    {ASPLOS 2022}
    {https://afuerst.github.io/assets/5B_0262_Fuerst.pptx}
    {https://www.youtube.com/watch?v=fvPAzienOTQ}

    \item \PresentationEntry{FaasCache: Keeping Serverless Computing Alive With Greedy-Dual Caching}
    {ASPLOS 2021}
    {https://afuerst.github.io/assets/ASPLOS-2021-pres.pptx}
    {https://www.youtube.com/watch?v=vpP5nROZpDM}

    \item \PresentationEntry{Cloud-scale VM-deflation for Running Interactive Applications On Transient Servers}
    {HPDC 2020}
    {https://afuerst.github.io/assets/HPDC-2020-pres.pptx}
    {https://www.youtube.com/watch?v=gFzaHkM\_1Tg}
  \end{itemize}
}

\newcommand{\iluvatar} {
  \ProjectEntry{\href{https://github.com/cos-in/iluvatar-faas}{\sysname~FaaS Control Plane}}
  {An open-source, fast, jitter-free control plane for Serverless function execution written in $\sim$23k lines of Rust.
  \sysname~provides a 3x reduction in overhead compared to popular open-sourced examples under normal load, and under high load has a 100x reduction in p99 latency.
  Additionally, it enables unique usability and extensibility designed to accelerate FaaS research.}
}

\newcommand{\faascache} {
  \ProjectEntry{FaasCache}
  {Introduced caching insights into the Function-As-A-Service paradigm.
  Enhanced the open source FaaS application OpenWhisk using Greedy-Dual caching.
  Reduced cold-start overhead for functions by up to 3x and can reduce constrained system resources by up to 30\%.
  These high cache reuse results allow for increased ability to serve functions and lower latency for users.}
}

\newcommand{\compucell} {
  \ProjectEntry{CompuCell3D Tissue Modeling Parallel Rendering}
  {CompuCell3D is a 3D modeling software for large-scale cellular, tissue, and biochemical simulation.
  The modeling steps used an OpenMP Cellular Potts Model algorithm, but the 3D rendering of cell states and positions was done serially.
  This project re-wrote the rendering code in $\sim$1000 lines of OpenMP C, achieving a near-linear scaling with the increase in threads.
  Overall, some simulations were accelerated by up to 50\% over the serial implementation.}
}

\newcommand{\Projects} {
  \NewPart{Projects}

  \iluvatar
  
  \faascache
  
  \compucell

  \ProjectEntry{Dynamically Typed Racket Compiler}
  {A scratch built compiler supporting a subset of statically typed and dynamically typed Racket.}
  
  \ProjectEntry{Tensorflow NanoParticle Simulator}
  {Implementation of the Lennard-Jones potential in a simulated cube and electrostatic forces of colliding ions in a confined nano-channel.
  THe simulator Achieved performance similar to MPI/C++ code performing the same simulation.}
  
  \ProjectEntry{Jae OS}{Just Another Educational Operating System. A port of the Kaya OS project to the new $\mu$ARM emulator. 
  Wrote the student guide and the canonical implementation of Jae OS.}
  
  \ProjectEntry{Kaya OS}{Wrote a complete operating system from scratch. 
  The final product, in addition to support a multitude of peripheral devices, successfully ran eight concurrent processes, each running in their own virtual address space.}
  
  \ProjectEntry{Parallel GIS Raster Calculator}{Developed a tool combining CPU based parallelism and NVIDIA's CUDA technology for GPU calculation for performing GIS raster calculations. 
  Achieved 2 - 5 times performance increase over traditional analysis tools due to GPU performance.}
  
  \ProjectEntry{Eagle Scout Project}{Installed commemorative plaques on veterans’ graves at local cemetery. 
  Led a group of 15 scouts to plan and accomplish this project.}
  
}

\newcommand{\Courses} {
  \NewPart{Course Work}

  \begin{multicols}{2}
    \begin{itemize}
      \setlength\itemsep{0em}
      \item Engineering Cloud Computing
      \item Engineering Distributed Systems
      \item Graph Analytics
      \item Deep Learning Systems
      \item Engineering Compilers
      \item Programming Languages
    \end{itemize}
  \vfill\null
  \columnbreak
    \begin{itemize}
      \setlength\itemsep{0em}
      \item Engineering Operating System
      \item Simulating Nanoscale Systems
      \item High Performance Computing
      \item Computational Modeling for Virtual Tissues % Computational Modeling Methods for Virtual Tissues
      \item Databases
      % \item Data Structures \& Algorithms
    \end{itemize}
  \vfill\null
  \end{multicols}
} 

\newcommand{\ProgrammingSkills} {
  \SkillsEntry{Programming}{Debugging, Problem Solving, Code Optimization, Git, Agile}
}
\newcommand{\LanguagesSkills} {
  \SkillsEntry{Languages}{Rust, Python, C, C++, C\#, Bash, PowerShell, SQL, \LaTeX} % , Java, Scheme
}
\newcommand{\InfrastructureSkills} {
  \SkillsEntry{Infrastructure}{Kubernetes, Docker, Redis, Octopus Deploy, Ansible, AWS, Azure}
}
\newcommand{\TechnologiesSkills} {
  \SkillsEntry{Technologies}{Linux, KVM, GPUs, GDB, OpenMP, MPI, Tensorflow, SQL Server}
}
\newcommand{\AdditionalSkills} {
  \SkillsEntry{Additional}{Research, DevOps, Presentations, Technical Writing, Agile, Git, Debugging} % , Documentation
  }
\newcommand{\Skills} {
  \NewPart{Skills}

  \ProgrammingSkills
  \LanguagesSkills
  \InfrastructureSkills
  \TechnologiesSkills
  % \SkillsEntry{Technologies}{Ansible, Linux, KVM, Octopus Deploy, GDB, OpenWhisk, Tensorflow, OpenMP, MPI}
  %\SkillsEntry{OS}{Linux, Windows, Mac; various editions}  
}

\newcommand{\Awards} {
  \NewPart{Awards}

  \AwardEntry{Reserve Champion, Baked Goods Division}{}{Monroe County Fair 2023}
  \AwardEntry{HPDC Travel Grant}{}{Travel grant to HPDC 2023}
  \AwardEntry{ACM Travel Grant}{}{Travel grant to ASPLOS 2022}
  \AwardEntry{John F. Niehaus Scholarship}{}{Xavier University}
  \AwardEntry{John F. Niehaus Award}{}{Xavier University}  
  \AwardEntry{Eagle Scout}{}{Boy Scouts of America}
  \AwardEntry{National Honors Society}{}{Medina High School}
  \AwardEntry{National Technical Honors Society}{}{Medina County Career Center}
}

\newcommand{\Activities} {
  \NewPart{Activities}

  \AwardEntry{Parish Pastoral Council, St. Paul's Catholic Center}{}{Member, 2023-2024}
  \AwardEntry{Young Adult Ministry, St. Paul's Catholic Center}{}{Vice Chair, 2023-2024}
  \AwardEntry{}{}{Member, 2022-2023}
  \AwardEntry{Computer Science Club, Xavier University}{}{Vice President, 2016-2017}
  \AwardEntry{}{}{Treasurer, 2015-2017}
  \AwardEntry{Dean’s Advisory Council, Xavier University}{}{Member, 2015 – 2017}
}

