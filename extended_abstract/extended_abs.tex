\documentclass[pageno]{jpaper}
\usepackage[normalem]{ulem}

\begin{document}

\newcommand{\quotes}[1]{``#1''}
\newcommand{\sysname}{Il\'{u}vatar} % \xspace}
\newcommand{\todo}[1]{\textcolor{red}{TODO: #1}}
% https://tex.stackexchange.com/a/7045
\newcommand*\circled[1]{\tikz[baseline=(char.base)]{
  \node[shape=circle,draw,inner sep=2pt] (char) {#1};}}

  \newcommand{\D}{\emph{D}}
\newcommand{\Dmax}{\emph{D\_max}}
\newcommand{\T}{\emph{T}}
\newcommand{\VT}{\emph{VT}}
\newcommand{\GlobVT}{\emph{Global\_VT}}
\newcommand{\FinVT}{\todo{REMOVE: finish\_vt}}
% \newcommand{\QName}{MQ-FLQ}
% \newcommand{\QNameFull}{Multi-Queue Fair \& Local Queuing}
\newcommand{\QName}{MQFQ-Sticky}
\newcommand{\QNameFull}{\QName}
\newcommand{\batch}{\texttt{Batch}}
\newcommand{\naive}{\texttt{FCFS Naïve}}
\newcommand{\fcfs}{\texttt{FCFS}}

\newcommand{\myfigspace}[0]{-0.3cm}
\newcommand{\bigfigspace}[0]{-0.9cm}
\newcommand{\captionspace}[0]{-0.3cm}
\newcommand{\subsecspace}[0]{-0.15cm}
\newcommand{\largesubsecspace}[0]{-0.42cm}

\newcommand{\prat}[1]{\textcolor{red}{(Prateek: #1)}}
\newcommand{\alex}[1]{\textcolor{blue}{Alex: #1}}


\newcommand{\funcname}[1]{\texttt{#1}}
\newcommand{\mhead}[1]{\textbf{#1.}}

\newcommand{\squishlist}{
 \begin{list}{$\bullet$}
  { \setlength{\itemsep}{0pt}
     \setlength{\parsep}{3pt}
     \setlength{\topsep}{3pt}
     \setlength{\partopsep}{0pt}
     \setlength{\leftmargin}{1.5em}
     \setlength{\labelwidth}{1em}
     \setlength{\labelsep}{0.5em} } }
	
\newcommand{\squishlisttwo}{
 \begin{list}{$\bullet$}
  { \setlength{\itemsep}{0pt}
     \setlength{\parsep}{0pt}
    \setlength{\topsep}{0pt}
    \setlength{\partopsep}{0pt}
    \setlength{\leftmargin}{2em}
    \setlength{\labelwidth}{1.5em}
    \setlength{\labelsep}{0.5em} } }

\newcommand{\squishend}{
  \end{list}  }

\def\myfontsize{\fontsize{12pt}{12pt}\selectfont}

\ifdefined\tiny
\renewcommand{\tiny}{\myfontsize}
\else
\newcommand{\tiny}{\myfontsize}
\fi

\ifdefined\scriptsize
\renewcommand{\scriptsize}{\myfontsize}
\else
\newcommand{\scriptsize}{\myfontsize}
\fi  

\ifdefined\footnotesize
\renewcommand{\footnotesize}{\myfontsize}
\else
\newcommand{\footnotesize}{\myfontsize}
\fi  

\ifdefined\small
\renewcommand{\small}{\myfontsize}
\else
\newcommand{\small}{\myfontsize}
\fi  

\ifdefined\large
\renewcommand{\large}{\myfontsize}
\else
\newcommand{\large}{\myfontsize}
\fi  

\ifdefined\Large
\renewcommand{\Large}{\myfontsize}
\else
\newcommand{\Large}{\myfontsize}
\fi  

\ifdefined\LARGE
\renewcommand{\LARGE}{\myfontsize}
\else
\newcommand{\LARGE}{\myfontsize}
\fi  

\ifdefined\Huge
\renewcommand{\Huge}{\myfontsize}
\else
\newcommand{\Huge}{\myfontsize}
\fi  

\ifdefined\HUGE
\renewcommand{\HUGE}{\myfontsize}
\else
\newcommand{\HUGE}{\myfontsize}
\fi  

\title{Serverless Control Planes for Orchestration of Cloud Resources \\ \textbf{Extended Abstract}}
\author{Alexander Fuerst}

\date{}

\begin{comment}
The SPEC Research Group promotes research in quantitative system evaluation and analysis
both with classical performance metrics – such as response time, throughput, scalability and efficiency,
as well as other extra-functional system properties included under the term dependability – 
such as availability, reliability, and security. Contributions of interest span the design of metrics
for system evaluation as well as the development of methodologies, techniques and tools for
measurement, load testing, profiling, workload characterization, dependability and efficiency evaluation of computing systems.

A separate one-page extended abstract. 
You can take the current abstract and add a few more sentences about the technical novelty, results, etc., 
since this will be for a performance-engineering audience  
\end{comment}

\maketitle

\thispagestyle{empty}

\section{Motivation}


Cloud computing is comprised of a variety of services and abstractions to reduce the complexity of building and running applications.
Function as a Service (FaaS) is a unique abstraction amongst the many: user supplied code entirely managed by the cloud provider.
Serverless control planes face unique challenges caused by the highly heterogeneous workloads predominant.
Users run machine learning inference, web services, multimedia analysis, and even scientific computing, to the tune of billions of invocations per day.
These, in turn, vary widely in resource usage, execution time, and frequency of invocation, which the control plane must handle efficiently.
A typical function invocation runs for less than a second and uses only 128 MB of memory, after invocation, these resources can sit idle indefinitely.
Handling functions like classical cloud applications can cause significant latency delays and tremendous wasted resources.

The diverse platforms serverless has been proposed to run on vary in scale, prupose, and even hardware design.
Edge~\cite{hall_execution_2019}, public cloud, and heterogeneous~\cite{du2022serverless} scenarios have different capabilities and resource requirements.
Existing open-source platforms rely on complex and high-overhead systems like Docker or Kubernetes~\cite{openwhisk,openfaas} which do not fit on low-powered devices and perform poorly at FaaS scale.
Each invocation must run inside an isolation mechanism, which can take several seconds and add orders of magnitude of time to latency.
The large tech stacks of these designs wastes precious resources, adds 100s of milliseconds to invocation overhead, and are ill-suited to fast-paced research advancements.
% It must also handle problems like cold start overheads neccessitated by executing invocations within isolated containers, and resource underutilization during idle periods.

\section{Insights}

A key insight of this thesis is that existing open-source FaaS platforms are neither performant at scale nor launchpads for advanced research across the many serverless topics. 
Current server-full mechanisms for hosting software and managing resources poorly support the needs of the emerging FaaS ecosystem.
Centralized load balancing or scheduling designs, such as is found in OpenWhisk~\cite{openwhisk}, cause unnecessary overhead or fail to scale with FaaS workloads, often both.
A platform meeting these performance goals that can run using limited system resources, is easily extensible, and runnable on devices from the edge-to-cloud continuum is needed.


\section{Key Contributions}

This thesis addresses performance and resource management challenges by developing novel system designs and tailoring algorithms to handle the new workloads and platforms of FaaS.
I detail several algorithms that tackle poor resource allocation and load scheduling at both individual server and cluster levels.
The key finding of orchestration at both levels is the reliance on function characteristics in decision-making.
The centerpoint of this thesis is a redesigned control plane, called \sysname~\cite{fuerst2023iluvatar}.
\sysname~handles requests with less than \textit{2 milliseconds} of platform overhead and reduces latency spikes by 100x found in existing open-source systems.
This is primarily accomplished by disaggregating scheduling decisions to avoid contention between distributed services of the control plane.
Finally, I leverage the high degree of resource control made possible by this new system to integrate GPU acceleration into the serverless ecosystem.
This uses a variety of novel mechanisms to minimize overhead despite limited device resources, boosting performance by several orders of magnitude over baseline solutions.
Altogether, the contributions of this thesis serverless improve latency by 75\% and cluster resource utilization by up to 20x.

\sysname~was written in 20k lines of Rust code, with the goal of making new research easier at both ends of development and experimentation.
This thesis shows two examples of this, first implementing alternate container isolation mechanisms that are faster than Docker, and second comparing GPU scheduling policies for complex invocation workloads.
Results captured by the built-in workload generation tools are designed to be easily parseable for post-analysis writeup of results.
% Previous research that has done wither of these tasks have built entire system designs from the ground up, making difficult 

\bibliographystyle{plain}
\bibliography{../refs.bib,
../chrlu/faaslb-osdi22/camready/related.bib,
../chrlu/faaslb-osdi22/camready/faas-asplos.bib,
../chrlu/faaslb-osdi22/camready/faas.bib,
../chrlu/faaslb-osdi22/camready/lb.bib,
../chrlu/qos/paper/lb.bib,
../chrlu/qos/paper/faas.bib,
../chrlu/qos/paper/faas-asplos.bib,
../chrlu/qos/paper/differentiated.bib,
../iluvatar/camera-ready/efaas.bib,
../iluvatar/camera-ready/faas.bib,
../iluvatar/camera-ready/faas-asplos.bib,
../faascache/faas-keepalive-20/asplos_camready/faas.bib,
../mqfq-final/paper/gpu-q-faas.bib}

\begin{comment}
ASPLOS'21 will be piloting the submission of {\it Extended
Abstracts}. Your extended abstract — inspired by the model used for
IEEE Micro Top
Picks — should be {\it two pages long}, and it will be submitted separately
from your main paper. The deadline for the extended abstract and the
full paper will be identical. Except for the page limit, \textbf{all other
formatting and anonymity requirements are identical} to those for full
papers: no names or affiliations should be present, and citations should
be anonymized if appropriate. Extended abstracts should be self-contained, though they may
contain references to the full paper.

We \emph{recommend}---but do not require--- that you use
the following organization for your abstract. 
Sections~\ref{sec:motivation} through \ref{sec:key-contributions}
should be a summary of your full paper. Section~\ref{sec:motivation}
motivates the paper; Section~\ref{sec:limitations} describes
limitations of the state of the art, if applicable;
Section~\ref{sec:key-insights} presents the key new insight or
insights of the paper; 
Section~\ref{sec:main-artifacts} presents the main artifacts described
in the paper;  Section~\ref{sec:key-contributions} summarizes the key
results and technical contributions of your paper. Finally,
Section~\ref{sec:why-asplos} should explain why the paper is suitable
for ASPLOS, and Section~\ref{sec:citation} should \emph{anonymously} state what its
citation would be for a ten year test-of-time award. In
Section~\ref{sec:revisions}, you may optionally include a paragraph
describing how your paper has been revised, if it was previously
submitted to another conference.

The extended abstracts must be submitted in printable PDF format and should contain a
{\bf maximum of 2 pages} of single-spaced two-column text, {\bf not
  including references}.  You may include any number of pages for
references, but we suggest you limit your bibliography to
only the most relevant references. The extended
abstracts should use the same formating as the regular papers. If you are using
\LaTeX~\cite{lamport94} to typeset your extended abstract, then we suggest that
you use \href{https://asplos-conference.org/wp-content/uploads/2020/06/asplos21-templates.zip}{this template}.
If you use a different
software package, then please adhere to the
\href{https://asplos-conference.org/wp-content/uploads/2020/06/asplos21-paper-template.pdf}{formatting guidelines}
for ASPLOS papers.

%The \href{ https://asplos-conference.org/wp-content/uploads/2020/06/asplos21-extended-abstract-template.pdf}{sample file} for the
%extended abstract includes guidelines the information your abstract
%should include. 


%The references section of your extended abstract will not count
%towards the two page limit. We suggest you limit your bibliography to
%only the most relevant references~\cite{lamport94}.
The extended abstract should not have an abstract. Start with Section~\ref{sec:motivation}.

\section{Motivation}
\label{sec:motivation}


\begin{itemize}
\item What is the problem your work attacks? Be specific.
\item Why is it an important problem?
\end{itemize}

\vspace{1em}

\noindent
Articulate the importance of this problem to the broader ASPLOS
community, using as little jargon as possible. \emph{Be specific}
about the problem you are addressing; it should be the one that your
paper directly addresses.

\section{Limitations of the State of the Art}
\label{sec:limitations}

\begin{itemize}
\item What is the state of the art in this topic today (if any)?
\item What are its limits?
\end{itemize}

\section{Key Insights}
\label{sec:key-insights}

\begin{itemize}
\item What are the one or two key new insights in this paper?
\item How does it advance the state of the art?
\item What makes it more effective than past approaches?
\end{itemize}

\section{Main Artifacts}
\label{sec:main-artifacts}

\begin{itemize}
\item What are the key artifacts presented in your paper: a
  methodology, a hardware design, a software algorithm, an
  optimization or control technique, etc.?
  \item How were your artifacts implemented and evaluated? 
\end{itemize}

\section{Key Results and Contributions}
\label{sec:key-contributions}

\begin{itemize}
  \item What are the most important \emph{one or two} empirical or theoretical
    results of this approach?
  \item What are the contributions that this paper makes to the state of the
    art? List them in an \texttt{itemize} section. Each contribution should be no more than a few sentences long.
  \item Clearly describe its advantages over past work, including how it overcomes their limitations.
\end{itemize}


\section{Why ASPLOS}
\label{sec:why-asplos}

ASPLOS emphasizes multidisciplinary research; explain how this
  paper emphasizes synergy of \emph{two or more ASPLOS areas}: architecture,
  programming languages, operating systems, and related areas (broadly
  interpreted).

\noindent
If you are unsure whether your paper falls within the scope of ASPLOS,
please check with the program chairs -- ASPLOS is a broad,
multidisciplinary conference and encourages new topics.

\section{Citation for Most Influential Paper Award}
\label{sec:citation}

Provide the citation for your paper if it won a Most Influential
Paper award. You can find example citations
on the \href{https://www.sigops.org/awards/hof/}{SIGOPS Hall of Fame}
and \href{https://www.sigplan.org/Awards/PLDI/}{PLDI most influential
  paper list}.  Limit the citation to 1-3 sentences
\emph{Recall that your citation here must be anonymous}; do not include names or affiliations.

%  \url{https://rb.gy/hd1hms}).

\section{Revisions}
\label{sec:revisions}

 \emph{Optional:} Describe how this paper has been revised, if it was previously submitted to another conference.

 
\pagebreak
\bibliographystyle{plain}
\bibliography{references}
\end{comment}

\end{document}

